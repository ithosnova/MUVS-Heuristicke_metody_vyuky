
\chapter{Metody výuky v pedagogice}

Metody jsou v obecném pojetí definovány jako způsoby nebo soustavné postupy, které v dané problematice vedou k dosažení vytyčeného cíle. Ve smyslu vyučovacích metod se jedná o záměrný způsob (postup) činnosti učitele či žáka směřující k dosažení výchovně vzdělávacích cílů. \cite{dvoracek2000,vanecek2016}\\

V současné době mezi nejznámější dělení výukových metod patří
komplexní dělení dle Josefa Maňáka, které člení výukové metody:
 z hlediska pramene poznání = didaktický aspekt, z hlediska aktivity a samostatnosti žáků = psychologický aspekt, z hlediska fází výuky = procesuální aspekt, z hlediska myšlenkových operací = logický aspekt, z hlediska výukových forem a prostředků = organizační aspekt a z hlediska interakce a komunikace žáků s učitelem a žáků mezi sebou navzájem = interaktivní aspekt. Josef Maňák s Vlastimilem Švecem publikovali také kombinované dělení výukových metod, které výukové metody rozlišuje podle kritéria stupňující se složitosti edukačních vazeb. Toto dělení člení metody do tří skupin: {\it klasické výukové metody, aktivizující výukové metody a komplexní výukové metody.} \cite{slavik2012,zormanova2012}
 
    \begin{itemize}
        \item \textbf{Klasické metody} \\
            Jsou charakteristické frontální výukou, kde dominantní roli hraje učitel. Mají dlouhou historii a stále se hojně využívají.
            
            \begin{enumerate}
                \item Metody slovní
                    \begin{itemize}
                        \item[a)] Monologické (přednáška, vysvětlování, výklad, instruktáž)
                        \item[b)] Dialogické (rozhovor, diskuze, dramatizace)
                        \item[c)] Metody písemných prací
                        \item[d)] Metody práce s učebnicí, knihou
                    \end{itemize}{}
                    
                \item Metody názorně demonstrační
                    \begin{itemize}
                        \item[a)] Pozorování předmětů a jevů
                        \item[b)] Předvádění obrazů a předmětů, pokusů, činností
                        \item[c)] Projekce statická a dynamická
                    \end{itemize}{}

                \item Metody praktické
                    \begin{itemize}
                        \item[a)] Nácvik pohybových apracovních dovedností
                        \item[b)] Žákovy pokusy a laboratorní činnosti
                        \item[c)] Pracovní činnosti v dílnách, na pozemcích
                        \item[d)] Grafické a výtvarné práce
                    \end{itemize}{}
                    
            \end{enumerate}
            
        \item \textbf{Aktivizující metody} \\
            Měli by působit na žáky stimulačně a podporovat jejich tvořivé myšlení, obvykle jsou založeny na řešení problémových situací a úloh.
            
            \begin{enumerate}
                \item Diskuzní metody
                \item Metody heuristické, řešení problémů
                \item Metody situační
                \item Didaktické hry
            \end{enumerate}{}
            
        \item \textbf{Komplexní metody} \\
            Jedná se o ucelenou kombinaci a propojení několika prvků didaktického systému pro danou výukovou metodu - propojení metody, organizační formy výuky, didaktických prostředků nebo životních situací.
        
            \begin{enumerate}
                \item Frontální výuka
                \item Skupinová a kooperativní výuka
                \item Partnerská výuka
                \item Individuální a individualizovaná výuka, samostatná práce žáků
                \item Kritické myšlení
                \item Brainstorming
                \item Projektová výuka
                \item Výuka dramatem
                \item Otevřené učení
                \item Učení v životních situacích
                \item Televizní výuka
                \item Výuka podporovaná počítači
                \item Sugestopedie a superlearning
                \item Hypnopedie
            \end{enumerate}{}
    \end{itemize}{}
 
 




