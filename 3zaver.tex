\chapter{Závěr}

Heuristické metody výuky a problémové úlohy patří mezi aktivizující metody výuky, které vedou k rozvoji tvořivosti a samostatnosti žáků či studentů. Tyto metody mají jistě svoje místo ve všech stupních vzdělávání. Ačkoliv to zde nebylo uvedeno, tak vzdělávání nejmenší dětí má značné heuristické prvky. Velké místo u dětí zaujímají názorně demonstrační metody, ale každé dítě je takový malý badatel objevující svět. Aktivizující metody výuky obecně velmi dobře dodržují didaktické zásady názornosti, aktivnosti a často také zásadu spojení teorie s praxí. Obavy z jejich zařazení do výuky jsou v některých případech oprávněné, ale v mnoha dalších jde jen domnělé překážky. Heuristiké rozhovory mohou ve velmi krátkém čase ověřit směřování k výchovně vzdělávacím cílů, zlepšit pozornost žáků a motivovat je k dalšímu vzdělávání. Tyto metody jsou také velmi dobrým základem k sebevzdělávání u starších žáků a celoživotnímu pozitivnímu vztahu k učení a objevování. Tyto metody by měli být také součástí vysokoškolského vzdělávání. Pro tyto metody je jistě prostor na seminářích či cvičeních, ale velmi dobře mohou posloužit jako aktivizující formy výuky doplňující klasické přednášky.\\

Já sama se na heuristické metody dívám s obavou začínajícího učitele. Heuristické metody předpokládají od učitele, že bude pohotový, vnímavý, zvládne organizaci třídy, že bude kreativní a tvořivý. Myslím, že tyto metody jsou výzvou pro každého začínajícího pedagoga, ale jsou tím nejlepším, co můžeme žákům do výchovně vzdělávacího procesu přinést.\\

